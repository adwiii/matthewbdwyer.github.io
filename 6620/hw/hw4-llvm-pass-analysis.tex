\documentclass[12pt,letterpaper]{article}
\usepackage{fullpage}
\usepackage[top=2cm, bottom=4.5cm, left=2.5cm, right=2.5cm]{geometry}
\usepackage{amsmath,amsthm,amsfonts,amssymb,amscd}
\usepackage{lastpage}
\usepackage{enumerate}
\usepackage{fancyhdr}
\usepackage{mathrsfs}
\usepackage{xcolor}
\usepackage{graphicx}
\usepackage{listings}
\usepackage{hyperref}

\hypersetup{%
  colorlinks=true,
  linkcolor=blue,
  linkbordercolor={0 0 1}
}
 
\setlength{\parskip}{0.05in}

% Edit these as appropriate
\newcommand\course{CS 6620}
\newcommand\hwnumber{4}                  % <-- homework number

\pagestyle{fancyplain}
\headheight 35pt
\lhead{12 points}
\chead{\textbf{\Large Homework \hwnumber}}
\rhead{\course \\ assigned: Mar. 5, 2020\\ 
presentation dates: Apr. 09 and 14, 2020\\
}
\lfoot{}
\cfoot{}
\rfoot{}
\headsep 1.5em

\begin{document}
~

You are to select, study, and ``present'' an LLVM pass to the class.
You are free to select any pass that you like, except for a trivial
pass like ``Hello World''.
For your chosen pass you are expected to:
\begin{itemize}
\item Explain what the pass does both conceptually and through examples;
\item Relate the pass to the concepts, data structures and algorithms studies in the class;
\item Explain any dependencies of this pass on other parts of LLVM;
\item Illustrate the pass in action in processing C or TIP programs; and
\item If you choose an analysis pass, compare it to the closest corresponding TIP analysis. 
\end{itemize}

The first thing you should do on this assignment is to sign up for
a 25 minute presentation slot on Apr. 9 or Apr. 14.  There are
3 slots per class and they will be assigned on a first-come first-served
basis.  
You can sign up by sending the instructor an email with the 
team members and the selected pass.  
When presenters sign-up I will update the course web-page.

In preparing and presenting your 25 minute introduction 
to and explanation of your selected pass, you should cover all 
of the areas discussed above and 
provide a ``tutorial'' explanation of the pass to educate the class.

You will be graded on the organization of your presentation,
the quality of your presentation materials,
the clarity of your explanations,
the extent to which you use your 25 minutes appropriately (i.e., keeping
to the time limit, but also using the time appropriately),
and your ability to answer questions about your presentation. 

There are many good resources available online about how to prepare
and give a technical presentation.  This site is a nice example:
\url{https://homes.cs.washington.edu/~mernst/advice/giving-talk.html}

You may use any slide preparation system you would like, but you 
will submit a PDF copy of the slides you use in making your presentation.

\section*{Part 1 (4 points)}
You will submit a draft of your presentation materials 4 days before
your presentation slot so that you can get feedback from the instructor.
A complete set of materials will receive full credit, but a sketch or
limited outline will receive a reduced score.

This draft can be emailed to the instructor.

\section*{Part 2 (4 points)}
On or before April 15 you will submit the final set of materials you used
for the presentation.  These will be graded according to the above rubric 
in terms of organization, quality, clarity, etc.  

You will submit a PDF copy of your final presentation through collab.

\section*{Part 3 (4 points)}
The presentation will be evaluated as well in terms of how clearly
you present the material, how you address questions, and how you
engage the audience.


\end{document}


