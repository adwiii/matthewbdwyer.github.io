\documentclass[12pt,letterpaper]{article}
\usepackage{fullpage}
\usepackage[top=2cm, bottom=4.5cm, left=2.5cm, right=2.5cm]{geometry}
\usepackage{amsmath,amsthm,amsfonts,amssymb,amscd}
\usepackage{lastpage}
\usepackage{enumerate}
\usepackage{fancyhdr}
\usepackage{mathrsfs}
\usepackage{xcolor}
\usepackage{graphicx}
\usepackage{listings}
\usepackage{hyperref}

\hypersetup{%
  colorlinks=true,
  linkcolor=blue,
  linkbordercolor={0 0 1}
}
 
\setlength{\parskip}{0.05in}

% Edit these as appropriate
\newcommand\course{CS 6620}
\newcommand\hwnumber{3}                  % <-- homework number

\pagestyle{fancyplain}
\headheight 35pt
\lhead{14 points}
\chead{\textbf{\Large Homework \hwnumber}}
\rhead{\course \\ assigned: 10/16/2019\\ 
oral due: 10/30 - 11/6, 2019\\
written due: 11/6/2019
}
\lfoot{}
\cfoot{}
\rfoot{}
\headsep 1.5em

\begin{document}
~

You are to select, study, and ``present'' an LLVM pass to the class.
You are free to select any pass that you like, except for a trivial
pass like ``Hello World''.
For your chosen pass you are expected to:
\begin{itemize}
\item Explain what the pass does both conceptually and through examples;
\item Relate the pass to the concepts, data structures and algorithms studies in the class;
\item Explain any dependencies of this pass on other parts of LLVM;
\item Illustrate the pass in action in processing C or TIP programs; and
\item If you choose an analysis pass, compare it to the closest corresponding TIP analysis. 
\end{itemize}

This assignment must be completed working with a partner.

The first thing you should do on this assignment is to sign up for
a 15 minute presentation slot on Oct. 30, Nov. 4, or Nov. 6.  There are
4 slots per class and they will be assigned on a first-come first-served
basis.  When presenters sign-up I will update the course web-page.

\section*{Part 1 (7 points)}
Prepare and present a 15 minute introduction to and explanation of your
selected pass.  You should cover all of the areas discussed above and 
provide a ``tutorial'' explanation of the pass to educate the class.

Both team members are required to participate in the presentation.
You will be graded on the organization of your presentation,
the quality of your presentation materials,
the clarity of your explanations,
the extent to which you use your 15 minutes appropriately (i.e., being
on time),
and your ability to answer questions about your presentation. 

There are many good resources available online about how to prepare
and give a technical presentation.  This site is a nice example:
\url{https://homes.cs.washington.edu/~mernst/advice/giving-talk.html}

You may use any slide preparation system you would like, but you 
will submit a PDF copy of the slides you use in making your presentation.


\section*{Part 2 (7 points)}
Prepare a written presentation of your selected pass that covers all of
the areas discussed above.  Your written presentation should be well-organized 
and cover all of the areas described above.  The writeup need not be long,
e.g., it will be less than 10 pages, but it should be clear and comprehensive.

The writeup will be graded on the organization of the presentation,
the quality of the presentation (e.g., grammar), and
the clarity of the explanations.

You should cite related
work appropriately, e.g., when you relate the pass to concepts, data
structures, and algorithms make an effort to find an appropriate citation.

You will submit a PDF copy of your written presentation.

\end{document}


