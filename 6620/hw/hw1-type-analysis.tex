\documentclass[12pt,letterpaper]{article}
\usepackage{fullpage}
\usepackage[top=2cm, bottom=4.5cm, left=2.5cm, right=2.5cm]{geometry}
\usepackage{amsmath,amsthm,amsfonts,amssymb,amscd}
\usepackage{lastpage}
\usepackage{enumerate}
\usepackage{fancyhdr}
\usepackage{mathrsfs}
\usepackage{xcolor}
\usepackage{graphicx}
\usepackage{listings}
\usepackage{hyperref}

\hypersetup{%
  colorlinks=true,
  linkcolor=blue,
  linkbordercolor={0 0 1}
}
 
\setlength{\parskip}{0.05in}

% Edit these as appropriate
\newcommand\course{CS 6620}
\newcommand\hwnumber{1}                  % <-- homework number

\pagestyle{fancyplain}
\headheight 35pt
\lhead{8 points}
\chead{\textbf{\Large Homework \hwnumber}}
\rhead{\course \\ assigned: Sept. 4, 2019\\ due: $\le$23:59 EST Sept. 16, 2019 }
\lfoot{}
\cfoot{}
\rfoot{}
\headsep 1.5em

\begin{document}
This assignment can be completed individually or in a pair.  All assignments in the class can be completed this way, but you cannot repeat any pairing during the course.  Please clearly indicate the names of the people submitting the solution on the two attached files.

Complete the TIP type analysis by: a implementing the undefined behavior in\\
\texttt{tip.analysis.TypeAnalysis.visit}, explaining three distinct cases of the type analysis that implemented , and providing evidence that your implementation is correct.  More specifically, you should submit the following files to the CS6620 collab site for HW1.
\begin{enumerate}
\item \texttt{TypeAnalysis.scala} with all \texttt{// <--- Complete here} text replaced with functional scala code.
\item A PDF file \texttt{hw1.pdf} with three parts:
\begin{enumerate}
\item Describe three distinct cases of type analysis that you implemented; ``distinct'' means that the structure of the generated type constraints are not simple variants of one another, e.g., renamings of parameters. 
\item Provide evidence of the correctness of your type implementation.  For example, you might describe how you tested your implementation and why you think it is adequate.
\item Explain how the \texttt{TypeAnalysis} and \texttt{UnionFindSolver} operate
when processing the following simple TIP program: \texttt{poly(p) \{ return *p; \}}
\end{enumerate}
\end{enumerate}

\section*{Hints}
You will want to become familiar with the following elements of the TIP analysis implementation:
\begin{itemize}
\item \texttt{tip.ast.Ast} and \texttt{tip.types.Types}
\item the \texttt{TypeAnalysis} makes use of \texttt{tip.ast.DepthFirstAstVisitor} not once, but twice; to generate and solve constraints and then to collect up the inferred types for each identifier and expression
\item \texttt{tip.ast.DepthFirstAstVisitor} is an instantiation of the visitor pattern for the \texttt{tip.ast.Ast}; this is a powerful design pattern that appears frequently in program analyzers
\item the \texttt{map} function is a clean way of transforming a list.  A call to $l.map(f)$ will produce a new list by applying $f$ to each element of $l$, i.e., if $l_i$ is the $i$th element of $l$, then $f(l_i)$ is the $i$th element of $l.map(f)$.
\item the main task in this assignment is to figure out how to map AST nodes to constraints that either unify terms for an AST node with a type or unify terms for two AST nodes.
\end{itemize}


\end{document}


