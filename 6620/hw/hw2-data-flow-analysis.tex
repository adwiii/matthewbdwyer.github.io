\documentclass[12pt,letterpaper]{article}
\usepackage{fullpage}
\usepackage[top=2cm, bottom=4.5cm, left=2.5cm, right=2.5cm]{geometry}
\usepackage{amsmath,amsthm,amsfonts,amssymb,amscd}
\usepackage{lastpage}
\usepackage{enumerate}
\usepackage{fancyhdr}
\usepackage{mathrsfs}
\usepackage{xcolor}
\usepackage{graphicx}
\usepackage{listings}
\usepackage{hyperref}

\hypersetup{%
  colorlinks=true,
  linkcolor=blue,
  linkbordercolor={0 0 1}
}
 
\setlength{\parskip}{0.05in}

% Edit these as appropriate
\newcommand\course{CS 6620}
\newcommand\hwnumber{1}                  % <-- homework number

\pagestyle{fancyplain}
\headheight 35pt
\lhead{12 points}
\chead{\textbf{\Large Homework \hwnumber}}
\rhead{\course \\ assigned: Sept. 18, 2019\\ due: $\le$23:59 EST Oct. 2, 2019 }
\lfoot{}
\cfoot{}
\rfoot{}
\headsep 1.5em

\begin{document}
This assignment can be completed individually or in a pair.  All assignments in the class can be completed this way, but you cannot repeat any pairing during the course.  Please clearly indicate the names of the people submitting the solution on the two attached files.

\section*{Part 1}
Complete the TIP data flow analysis for live variables.
\begin{enumerate}
\item implement the ??? cases
\item provide evidence
\end{enumerate}

\section*{Part 2}
Implement a reaching definitions data flow analysis for TIP.
\begin{enumerate}
\item 
\item
\end{enumerate}

\section*{Submissino} 
Upload to the course collab site the following three files: \texttt{LiveVarsAnalysis.scala}, \texttt{ReachingDefAnalysis.scala}, and a PDF file describing how you designed the reaching definition implementation and how you validated both of the analyses.

\section*{Hints}
You will want to become familiar with the following elements of the TIP analysis implementation:
\begin{itemize}
\item 
\item 
\item 
\item 
\item 
\end{itemize}


\end{document}


