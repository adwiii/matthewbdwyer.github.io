\documentclass[12pt,letterpaper]{article}
\usepackage{fullpage}
\usepackage[top=2cm, bottom=4.5cm, left=2.5cm, right=2.5cm]{geometry}
\usepackage{amsmath,amsthm,amsfonts,amssymb,amscd}
\usepackage{lastpage}
\usepackage{enumerate}
\usepackage{fancyhdr}
\usepackage{mathrsfs}
\usepackage{xcolor}
\usepackage{graphicx}
\usepackage{listings}
\usepackage{hyperref}

\hypersetup{%
  colorlinks=true,
  linkcolor=blue,
  linkbordercolor={0 0 1}
}
 
\setlength{\parskip}{0.05in}

% Edit these as appropriate
\newcommand\course{CS 6620}
\newcommand\hwnumber{3}                  % <-- homework number

\pagestyle{fancyplain}
\headheight 35pt
\lhead{14 points}
\chead{\textbf{\Large Homework \hwnumber}}
\rhead{\course \\ team members\\ ~~\\~~~~~~~~~~~~\\~~~~~~~~~~~~}
\lfoot{}
\cfoot{}
\rfoot{}
\headsep 1.5em

\begin{document}
~

You are to select, study, and ``present'' an LLVM pass to the class.
You are free to select any pass that you like, except for a trivial
pass like ``Hello World''.
For your chosen pass you are expected to:
\begin{itemize}
\item Explain what the pass does both conceptually and through examples;
\item Relate the pass to the concepts, data structures and algorithms studies in the class;
\item Explain any dependencies of this pass on other parts of LLVM;
\item Illustrate the pass in action in processing C or TIP programs; and
\item If you choose an analysis pass, compare it to the closest corresponding TIP analysis. 
\end{itemize}

\section*{Part 1 (7 points)}
Grading rubric for oral presentation:
\begin{itemize}
\item (1 point) Both team members participate
\item (1 point) Used 15 minutes appropriately (neither too long nor too short)
\item (1 point) Clarity of oral presentation
\item (1 point) Clarity of slide organization
\item (1 point) Answered questions effectively
\item (1 point) Illustrate the pass with an example
\item (1 point) Related pass to concepts in class
\end{itemize}



\section*{Part 2 (7 points)}
Grading rubric for written presentation:
\begin{itemize}
\item (2 points) Clear conceptual presentation of pass
\item (1 point) Illustrate the pass with an example
\item (2 points) Related pass to algorithms, concepts, data structures in class
\item (1 point) Cited related work
\item (1 point) Grammatically correct and well-written
\end{itemize}

\end{document}


